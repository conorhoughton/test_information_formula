\documentclass{beamer}
\usepackage[latin1]{inputenc}
%\usetheme{Montpellier}
\usetheme{Boadilla}
%\usecolortheme[RGB={204,51,255}]{structure}
%\usecolortheme[named=purple]{structure}
\usecolortheme[RGB={62,128,62}]{structure}
%\definecolor{dark}{rgb}{0.3,0.15,0.3}
%\definecolor{light}{rgb}{0.8,0.6,0.8}
%\definecolor{reddish}{rgb}{.5,0.15,0.15}
\definecolor{dark}{rgb}{0.4,0.2,0.4}
%\definecolor{light}{rgb}{0.8,0.6,0.8}
\definecolor{reddish}{rgb}{.7,0.25,0.25}
\usepackage{graphicx}
\usepackage{pstricks}

\usepackage{tikz}
\usetikzlibrary{arrows,decorations.markings,positioning}
\usepackage{epstopdf}

\title[Mutual information for functions.]{Mutual information for functions, maybe even ERPs.}
\author{Conor Houghton}
\institute{CS, U Bristol}
\date{Frankfurt, May 2018}

\begin{document}

\maketitle


\begin{frame}{Pretend ERPs 1}
\color{reddish}
\begin{center}
\include{example_erp_events}
\end{center}
\color{black} \textbf{Fictive event related potentials.} The
\lq{}stimuli\rq{} are random 5-vectors of landmarks; an ERP is
produces by perturbing the landmarks, interpolating with splines and
adding noise. \textbf{B} shows responses to different stimuli.
\color{black}
\end{frame}


\begin{frame}{Pretend ERPs 2}
\color{reddish}
\begin{center}
\include{example_erps_event_sigma0}
\end{center}
\color{black} \textbf{Multiple trials to the same stimulus.} The
amount the landmark points are moved is determined by $\sigma$; for
\textbf{A} $\sigma=0.0$, for \textbf{B} $\sigma=2$.  \color{black}
\end{frame}



\begin{frame}{Results 1}
\color{reddish}
\begin{center}
\include{info_v_event_sigma_plot}
\color{black} \textbf{Estimated mutual information.} \color{black}
\end{center}
\end{frame}


\begin{frame}{Results 2}
\color{reddish}
\begin{center}
  \include{info_v_trial_n}
\color{black} \textbf{Estimated mutual information.} Here $\sigma=1$. \color{black}
\end{center}
\end{frame}


%% \begin{frame}{Neurons}
%% \color{reddish}
%% \begin{center}
%% \includegraphics[width=6cm]{Kat_600.png}
%% \end{center}
%% \color{black}
%% \begin{center}
%% A Biocytin labelled CA1 pyramidal cell from a P14 rat hippocampus.
%% \end{center}
%% \vfill
%% \color{gray}
%% \flushright{\small{By Katarina Kolaric\\\texttt{http://www.bristol.ac.uk/neural-dynamics/programme-details/gallery}}}
%% \color{black}
%%\end{frame}

\end{document}

